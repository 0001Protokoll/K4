\documentclass[12pt, letterpaper]{article}

 
\begin{document}
\section{Durchführung}

Zuerst wurde eine Kalibrieung mit einer Ammoniumcarbonatlösung durchgeführt. Hierfür wurde eine Verdünnungsreihe mit den Konzentrationen von $4 \cdot 10^{-3}$~M, $3 \cdot 10^{-3}$~M, $2 \cdot 10^{-3}$~M, $1 \cdot 10^{-3}$~M und $0.5 \cdot 10^{-3}$~M einer wässrigen Ammoniumcarbonatlösung angesetzt und deren Leitfähigkeit mit Konduktometer \textit{Cond 3210} in einem auf $25$°C temperierten Becherglas unter ständigem rühren gemessen.

Anschließend wurden $250$~ml einer $0.1$~M Harnstofflösung hergestellt, welche dann auf jeweils $100$~ml der folgenden Konzentrationen verdünnt wurde:
$0.5 \cdot 10^{-3}$~M, $1 \cdot 10^{-3}$~M, $1.5 \cdot 10^{-3}$~M, $2 \cdot 10^{-3}$~M, $3 \cdot 10^{-3}$~M, $4 \cdot 10^{-3}$~M, $6 \cdot 10^{-3}$~M, $8 \cdot 10^{-3}$~M,  $1 \cdot 10^{-2}$~M. Außerdem wurden $100$~ml einer wässrigen Ureaselösung mit der Konzentration $1$~g/L hergestellt.

Die Messung erfolgte indem $100$~ml der Harnstofflösungen unterschiedlicher Konzentration in dem Becherglas vorgelegt und auf $25$°C temperiert wurde. Sobald die Temperatur erreicht worden ist, wurde unter ständigem Rühren die Leitfähigkeit bestimmt und $10$~ml der Ureaselösung hinzugegeben. Für $3$~Minuten wurde alle $10$~Sekunden die Leitfähigkeit gemessen. 

Um den Einfluss, der Ureaselösung auf die Leitfähigkeit zu messen, wurden $5$~ml auf $55$~ml mit Wasser verdünnt und erneut die Leitfähigkeit gemessen. 

Zwischen allen Messungen wurden das Becherglas, der Rührfisch und die Messsonde gespült.



Es wurde grundsätzlich Millipore-Reinst-Wasser verwendet.











\end{document}