%\input{usepackage.tex}
%\begin{document}
\section{Durchführung}
Zur Untersuchung der Hydrolyse von Saccharose sollte durch ein Polarimeter die Änderung des Drehwinkels bei isothermer Prozessführung untersucht werden. Vorab wurde das Reaktionsgefäß auf Reaktionstemperatur geheizt und die Beleuchtung des Polarimeters (Natriumdampflampe) angeschaltet. Die Reaktionslösung als  Saccharose Lösung ($25 Gew.-\%$) wurde in einem Becherglas angesetzt und anschließend für vier durchzuführende Messreihen in je einem Reagenzglas a $20\,\si{mL}$ überführt. Ferner wurde für jedes dieser Reagenzgläser ein weiteres Reagenzglas mit $20\,\si{mL}$ einer 3-N HCL befüllt. Alle Reagenzgläser wurden ebenfalls im Thermostaten auf die Reaktionstemperatur erwärmt. Die Temperaturüberprüfung fand mit einem analogen Thermometer statt. Nach Erreichen der Meßtemperatur sind  $10 \,\si{mL}$ der Salzsäure in eine Reaktionslösung unter kräftigem Rühren zugegeben worden. Die Zeitmessung begann ab der Zugabe von ungefähr $5\,\si{mL}$ der Salzsäure und ende 22 Minuten später. Ab der zweiten Minute wurden alle 2 Minuten eine Winkelbestimmung durch das Polarimeter durchgeführt. Diese Prozedur wurde insgesamt vier mal für die Reaktionstemperaturen von $[40,35,30,25] \,\si{^oC}$ durchgeführt. Es resultieren somit 10 Messwerte pro einer der vier Messreihen, also insgesamt 40 Messwerte.
%\end{document}
