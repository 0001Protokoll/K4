%\input{usepackage.tex}
%\begin{document}

\section{Einleitung}
Biochemische Reaktionen, also solche, welche Reaktionen des Stoffwechsels lebender Zellen umfassen, unterliegen in den häufigsten Fällen einer katalytischen Beschleunigung. Dabei nutzt die Zelle eigens dafür konstruierte Proteine die sogenannten Enzyme als Katalysatoren. Bemerkenswert ist hierbei, dass jeder Reaktion des Metabolismus ein hierfür spezialisiertes Enzym bereitgestellt wird. Die Urease ist ein prominentes Beispiel für ein solches Protein. Spezieller handelt es sich hierbei um ein Metalloprotein. Sie ist überwiegen in solchen Organismen zu finden, die sich als sogenannte Destruenten durch Zersetzung von organischen Material ernähren und so unter anderen aus Harnstoff den für viele Organismen so wichtigen Stickstoff in Form von Ammonium bereitstellen. Die Urease katalysiert also folgende chemische Reaktion.
\begin{equation}
\ch{CO(NH_2)_2 +  H_2O -> 2 NH_3 + CO_2}
\end{equation}

bzw. in wässriger Umgebung 
\begin{equation}
\ch{CO(NH_2)_2 + 2 H_2O -> 2 NH_4^+ + CO_3^{2-}}
\end{equation}

Diese Reaktion eignet sich gut, um die Konzepte der Kinetik einer enzymatischen Reaktion zu veranschaulichen und wurde daher schon häufig als Labor-Modell für eine solche Untersuchung gewählt.
