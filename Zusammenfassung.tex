%%\documentclass[a4paper, 12pt]{scrreprt}

\documentclass[a4paper, 12pt]{scrartcl}
%usepackage[german]{babel}
\usepackage{microtype}
%\usepackage{amsmath}
%usepackage{color}
\usepackage[utf8]{inputenc}
\usepackage[T1]{fontenc}
\usepackage{wrapfig}
\usepackage{lipsum}% Dummy-Text
\usepackage{multicol}
\usepackage{alltt}
%%%%%%%%%%%%bis hierhin alle nötigen userpackage
\usepackage{tabularx}
\usepackage[utf8]{inputenc}
\usepackage{amsmath}
\usepackage{amsfonts}
\usepackage{amssymb}

%\usepackage{wrapfig}
\usepackage[ngerman]{babel}
\usepackage[left=25mm,top=25mm,right=25mm,bottom=25mm]{geometry}
%\usepackage{floatrow}
\setlength{\parindent}{0em}
\usepackage[font=footnotesize,labelfont=bf]{caption}
\numberwithin{figure}{section}
\numberwithin{table}{section}
\usepackage{subcaption}
\usepackage{float}
\usepackage{url}
%\usepackage{fancyhdr}
\usepackage{array}
\usepackage{geometry}
%\usepackage[nottoc,numbib]{tocbibind}
\usepackage[pdfpagelabels=true]{hyperref}
\usepackage[font=footnotesize,labelfont=bf]{caption}
\usepackage[T1]{fontenc}
\usepackage {palatino}
%\usepackage[numbers,super]{natbib}
%\usepackage{textcomp}
\usepackage[version=4]{mhchem}
\usepackage{subcaption}
\captionsetup{format=plain}
\usepackage[nomessages]{fp}
\usepackage{siunitx}
\sisetup{exponent-product = \cdot, output-product = \cdot}
\usepackage{hyperref}
\usepackage{longtable}
\newcolumntype{L}[1]{>{\raggedright\arraybackslash}p{#1}} % linksbündig mit Breitenangabe
\newcolumntype{C}[1]{>{\centering\arraybackslash}p{#1}} % zentriert mit Breitenangabe
\newcolumntype{R}[1]{>{\raggedleft\arraybackslash}p{#1}} % rechtsbündig mit Breitenangabe
\usepackage{booktabs}
\renewcommand*{\doublerulesep}{1ex}
\usepackage{graphicx}
\usepackage{chemformula}


 
%\begin{document}
\section{Zusammenfassung}
Zu untersuchen war in den getätigten Versuch die enzymkatalysierte Zersetzung von Harnstoff in Ammoniumcarbonat. Insbesondere sollten Eigenschaften der Reaktionskinetik über die Michaelis Menten Theorie studiert werden. Die Ergbenisse als $v_{max}$ enstprechend der maximalen Reaktionsgeschwindigkeit , $k_2$ als Übergangskonstante des zweiten irreversiblen Schritt der Reaktion sowie $K_M$ als Michealis Menten Konstante, wurden in folgender Tabelle dargstellt :
\begin{table}[H]
	\centering
	\label{Erg}
	\caption{Erhaltene experimentelle Werte sowie Gegenüberstellung zur Literatur}
	\renewcommand*{\arraystretch}{1.4}
	\begin{tabular}{C{0.1\linewidth}|C{0.25\linewidth}C{0.25\linewidth}}
				& experimenteller Wert & Literatur\cite{otto} \\
		$ v_{max}\,\, [\si{\frac{mol}{L\cdot s}}] $ & $(2.92 \pm 0.03) \cdot 10^{-5}$ &   \\
		$k_2\,\,\,\,\,\,\,\, [\si{\frac{1}{s}}]$ & $(142.5 \pm 14.6)\quad \quad$ &  $25 \cdot 10^3$\\
		$K_M\,\, [\si{\frac{mol}{L}}]$& $(7.87 \pm 0.12) \cdot 10^{-4}$& $2 \cdot 10^{-4}$\\
	\end{tabular}
\end{table}
Weder die Michaelis Menten Konstante, noch die Maximalgeschwindigkeit sind verträglich mit den Literaturwerten. Insbesondere ist die Abweigung des experimentell bestimmten Wechselzahl $k_2$ auffällig. Da $k_2$ direkt Proportional zu der Konzentration der Urease ist, sowie antiproportional zu der maximalen Reaktionsgeschwindigkeit gemäß Gleichung \ref{eq:k2} kann durch diesen Zusammenhang die Abweichung untersucht werden. Nehmen wir an, dass die Konzentration der Urease nicht stark fehlerbehaftet war, somit begründet sich der Fehler von $k_2$ besonders durch $\frac{1}{v_{max}}$. Ferner ist $\frac{1}{v_{max}} = n$ gerade der \textit{offset} gemäß Gleichung \ref{eq:lineweaverPlot}. Es folgt also, dass der Y-Achsenabschnitt, berechnet durch die lineare Regression, einen zu hohen Wert angenommen hat. Dies liegt genau dann vor, wenn die reziproken Anfangsgeschwindigkeiten gerade Werte um den Y-Achsenabschnitt annehmen, sowie die Steigung verhältnismäßig gering ist. Zurück zum Experiment heißt dies also, dass die linearen Regressionen der durchgeführten Messreihen in Abbildung \ref{Messreihe}, zum einen eine zu ähnliche Steigung und zum anderen eine zu steile Steigung besitzen. Dies kann daran liegen, dass wir von einem idealen System ausgehen, also von einer Reinheit von 100\% des verwendeten Harnstoffes sowie der Urease ausgehen. Ebenfalls gehen wir davon aus, dass die Urease sich nicht durch äußere Einflüsse zersetzt hat. 

%\end{document}


