%\input{usepackage.tex}

 
%\begin{document}
\section{Zusammenfassung}
Zu untersuchen war in den getätigten Versuch die enzymkatalysierte Zersetzung von Harnstoff in Ammoniumcarbonat. Insbesondere sollten Eigenschaften der Reaktionskinetik über die Michael Menden Theorie studiert werden. Die Ergbenisse als $v_{max}$ enstprechend der maximalen Reaktionsgeschwindigkeit , $k_2$ als Übergangskonstante des zweiten, nicht gleichgewichts Schritt der Reaktion sowie $K_M$ als Micheal Menden Konstante, wurden in folgender Tabelle dargstellt :
\begin{table}[H]
	\centering
	\label{Erg}
	\caption{Erhaltene experimentelle Werte sowie Gegenüberstellung zur Literatur}
	\renewcommand*{\arraystretch}{1.4}
	\begin{tabular}{C{0.1\linewidth}|C{0.25\linewidth}C{0.25\linewidth}}
				& experimenteller Wert & Literatur$^{[2]}$ \\
		$ v_{max}\,\, [\si{\frac{mol}{L\cdot s}}] $ & $(2.92 \pm 0.03) \cdot 10^{-3}$ &   \\
		$k_2\,\,\,\,\,\,\,\, [\si{\frac{1}{s}}]$ & $(6.34 \pm 0.07)\cdot 10^{-7}$ &  $25 \cdot 10^3$\\
		$K_M\,\, [\si{\frac{mol}{L}}]$& $(8.08 \pm 0.01) \cdot 10^{-4}$& $2 \cdot 10^{-4}$\\
	\end{tabular}
\end{table}
Fehlerdiskussion folgt ...

%\end{document}


